% !TeX root = ../main.tex

\chapter{实验结果与分析}
实验的主要目的是为了解决以下的问题:

(1) 学到的角色是否能够动态地变化,并且这种变化是为了适应环境?(章节~\ref{sec:dynamic-role})

(2) 本框架能否促进子任务的分化、专业化?即拥有相似任务的智能体在角色空间有相似的角色表征,但是有不同的子任务的智能体有不同的角色,并且这个角色在角色空间相差很远。(章节~\ref{sec:role-evolution})

(3) 这样的子任务分化能否提高多智能体强化学习的性能?(章节~\ref{sec:baseline-ablation})

(4) 角色在训练的过程中是怎样演化的?以及这种演化对性能的影响是怎样?(章节~\ref{sec:role-evolution})

本相关实验的视频能在网站上看到\footnote{\url{https://sites.google.com/view/romarl/}}。项目的代码已公开\footnote{\url{https://github.com/drdh/pymarl}}

\section{基准实验和消融实验}\label{sec:baseline-ablation}

\begin{table}[htb]
    \centering\small
    \caption{基准实验和消融实验算法}
    \label{tab:baselines}
    \begin{tabular}{cll}
      \toprule
        & 算法   & 描述                         \\
      \midrule
          & IQL   & Independent Q-learning \\
          & COMA  & Foerster et~al.~\cite{foerster2018counterfactual} \\
      相关  & QMIX  & Rashid et~al.~\cite{rashid2018qmix} \\
      算法  & QTRAN & Son et~al.~\cite{son2019qtran} \\
          & MAVEN & Mahajan et~al.~\cite{mahajan2019maven} \\
      \midrule 
             & ROMA-RAW & 无损失函数$\mathcal{L}_I$和$\mathcal{L}_D$ \\
      消融-A    & QMIX-NPS & 每个智能体都有单独的局部效用网络的QMIX \\
            & QMIX-LAR & 带有和ROMA相当数量参数的QMIX \\
       \midrule
            & $\mathcal{L}_{TD}$ & 仅有TD损失函数,同ROMA-RAW \\
      消融-B & $\mathcal{L}_{TD}+\mathcal{L}_I$ & 没有$\mathcal{L}_D$的ROMA \\
            & $\mathcal{L}_{TD}+\mathcal{L}_D$ & 没有$\mathcal{L}_I$的ROMA \\
      \bottomrule
    \end{tabular}
  \end{table}

  本实验采用的基准实验和消融实验如表~\ref{tab:baselines}所示,ROMA(Role-Oriented MARL)代指本项目算法。其中,相关算法(见图~\ref{fig:performance-baselines})是为了明确比较本算法和其他已有的算法的性能,可以看到本算法有相当优异的性能;消融-A实验中(见图~\ref{fig:performance-ablations-A}),ROMA-RAW是为了明确上一章所介绍的损失函数的有效性;QMIX-NPS是为了说明独立的效用函数并不能达到本项目算法的性能;QMIX-LAR是为了说明更多的参数不是本算法性能提升的主要原因;消融-B实验见图~\ref{fig:performance-ablations-B})是为了明确各个损失函数对性能的影响。

  测试的环境是星际争霸2微操作基准(SAMC~\cite{samvelyan2019starcraft}). 这个环境包括各种各样的地图,这些地图被分类成简单、难、很难几个级别。上面选取的环境都是难或者很难。为了评估准确,所有的实验都是使用了5的随机种子,结果的曲线是带有$95\%$的置信区间的。

\begin{figure*}
    \includegraphics[width=\linewidth]{figures/learning-curve/learning_curve.pdf}
    \caption{ROMA与相关算法的性能对比}\label{fig:performance-baselines}
\end{figure*}

\begin{figure*}
    \includegraphics[width=\linewidth]{figures/learning-curve/ablation-A.pdf}
    \caption{消融实验-A}\label{fig:performance-ablations-A}
\end{figure*}

\begin{figure*}
    \includegraphics[width=\linewidth]{figures/learning-curve/ablation-B.pdf}
    \caption{消融实验-B}\label{fig:performance-ablations-B}
\end{figure*}
  
\section{角色的动态变化}\label{sec:dynamic-role}

\begin{figure*}
    \centering
    \includegraphics[height=0.24\linewidth]{figures/dynamic/10m_vs_11m-g1.pdf}\hfill
    \includegraphics[height=0.24\linewidth]{figures/dynamic/10m_vs_11m-g2.pdf}\\
    \subfigure[$t$=$1$, 角色的特征是智能体的位置]{\includegraphics[width=0.48\linewidth]{figures/dynamic/10m_vs_11m-r1.pdf}}\hfill
    \subfigure[$t$=$8$, 角色的特征是智能体的血量]{\includegraphics[width=0.48\linewidth]{figures/dynamic/10m_vs_11m-r2.pdf}}\\

    \includegraphics[height=0.24\linewidth]{figures/dynamic/10m_vs_11m-g3.pdf}\hfill
    \includegraphics[height=0.24\linewidth]{figures/dynamic/10m_vs_11m-g4.pdf}\\ 
    \subfigure[$t$=$19$, 角色的特征是智能体是否存活及血量]{\includegraphics[width=0.48\linewidth]{figures/dynamic/10m_vs_11m-r3.pdf}}\hfill
    \subfigure[$t$=$27$, 角色的特征是智能体是否存活及血量]{\includegraphics[width=0.32\linewidth]{figures/dynamic/10m_vs_11m-r4.pdf}}
    \caption{在一局里角色的动态变化}\label{fig:dynamic_role-10m_vs_11m}
\end{figure*}

为了回答角色是否是为了使用环境动态变化的,这里展示几张由本算法执行的截图,使用的环境是星际争霸2微操作环境(SMAC), 选用的地图是$\mathtt{10m\_vs\_11m}$, 在这个环境里,操控10个Marines对抗11个敌对Marines. 正如图~\ref{fig:dynamic_role-10m_vs_11m}所示,尽管观测里包含很多的信息,比如智能体的位置、血量、护盾值、同盟信息、敌人信息等等,但是角色编码器还是学会了集中注意力到部分关键的信息,并且这种注意力会随着环境的改变,动态地变化,这就是所谓的角色的动态变化。

具体来说,在刚开始$t$=$1$的时候,智能体们需要组织成为一个凹面的弧形,这样能最大化它们对敌人的射击范围,使得每个智能体都能射击到敌人最前面的那一行。可以从角色空间看到的是智能体角色的排布是根据它们的相对位置,如此根据这样的特化的策略能更快地形成上面提到的弧形站位。在战役的中间,一个很有用的策略是保护受伤严重的智能体,让它们后撤,这样能最大限度地保存战力,消灭敌人。可以从图中看到,在$t$=$8$, $19$, $27$的时候,智能体的角色决定于它自己的血量,血量高的智能体角色和其他智能体的角色相距很远。这样的表征会导致不同的策略:也就是血量高的智能体会向前移,吸引更多的火力,而其他的血量低的会后撤,从后方攻击。与此同时,也能发现同样的角色会因为位置相近而聚集在一起,比如智能体$3$和$8$在$t$=$19$的时候就聚在一起了。对应的策略是拥有不同角色的智能体会不断变换地吸引火力。除此之外,也能看到死亡的智能体会聚集在一起,并且数量不断增多,本质上来说,死亡是血量为$0$的极端情况。

这个结果就说明,本算法能够学到动态的角色,并且角色会依据自己观测到的子任务(比如前进、后撤)自动聚集,这一点也和上一章提出的目标函数的目的一致。

\section{角色的表征分析}\label{sec:role-representation}
\begin{figure*}
  \centering
  \includegraphics[width=0.32\linewidth]{figures/various-roles/various_roles-g1.pdf}\hfill
  \includegraphics[width=0.32\linewidth]{figures/various-roles/various_roles-g2.pdf}\hfill
  \includegraphics[width=0.32\linewidth]{figures/various-roles/various_roles-g3.pdf}
  \subfigure[策略:牺牲Zealots 9和7来消除Banelings的爆炸伤害。]{\includegraphics[width=0.32\linewidth]{figures/various-roles/various_roles-r1.pdf}\label{fig:various_roles-6s4z_vs_10b30z}}\hfill
  \subfigure[策略:快速形成一个凹型的站位。]{\includegraphics[width=0.32\linewidth]{figures/various-roles/various_roles-r2.pdf}\label{fig:various_roles-27m_vs_30m}}\hfill
  \subfigure[策略:绿色的Zerglings躲起来,然后Banelings用爆炸消灭大部分的敌人。]{\includegraphics[width=0.32\linewidth]{figures/various-roles/various_roles-r3.pdf}\label{fig:various_roles-6z4b}}
  \caption{角色的表征}\label{fig:various_roles}
  \note{地图$\mathtt{6s4z\_vs\_10b30z}$, $\mathtt{27m\_vs\_30m}$,和$\mathtt{6z4b}$学到的角色表征 (展示的是角色$\bm{\mu}_{\rho_i}$的均值,没有使用任何的降维)}
\end{figure*}

为了解释本算法的有效性,用图~\ref{fig:various_roles}来展示角色在角色空间的表征。注意到角色表示的是自动发现的胜利策略在隐层空间的表征、下面逐一说明。

在地图$\mathtt{6s4z\_vs\_10b30z}$中,见图~\ref{fig:various_roles-6s4z_vs_10b30z}, 本算法学会了牺牲Zealots 9和7来杀死所有的敌对Banelings(爆炸的Banelings自己也会死)。具体来说,Zealots 9和7会一个接着一个地向前,然后和敌人同归于尽,但是其他智能体会远离,等待直到所有的Banelings爆炸结束。图~\ref{fig:various_roles-6s4z_vs_10b30z}就是完成第一个子任务时,9号智能体牺牲时的角色表征,可以看到它的角色和其他的相距很远,这就是意味着9的策略和其他的非常不一样,从具体的回放截图也可以证实这一点。

对于地图$\mathtt{27m\_vs\_30m}$, 胜利策略是在战斗开始之初,交火之前,快速形成一个凹型的站位,图~\ref{fig:various_roles-27m_vs_30m}正是诠释了这一点,这里展示的是$t$=$1$的时刻的角色表征,智能体们正准备形成攻击站位。从角色空间也可以看到角色按照的是智能体的相对位置在聚集,这样的不同的表征会导致不同的移动策略,继而能迅速形成凹型站位,并且不发生冲突。

对于地图$\mathtt{6z4b}$, 胜利策略是Zerglinss 4和5以及Banelings杀掉其他大部分的敌人,这里是利用Banelings的爆炸。同时,Zerglings 6-9会躲远,直到爆炸结束,然后再去杀掉生下来的敌人。图~\ref{fig:various_roles-6z4b}显示的是在爆炸之前的角色表征,可以看到角色正是按照发现的子任务在聚集。

通过这些结果的证实,可以得出结论的是,本算法会自动地分解任务,并且能学到足够通用的角色,每个角色承担这样一个子任务。

\section{角色的演化与涌现}\label{sec:role-evolution}

\subsection{异质环境中角色的演化与涌现}
\begin{figure*}
  \centering
  \includegraphics[width=\linewidth]{figures/evolution/evolution_MMM2.png}
  \caption{角色在$\mathtt{MMM2}$中的演化与涌现}
  \label{fig:role_evolution-heter}
  \note{展示的是$t$=$1$的角色分布的$\bm{\mu}_{\rho_i}$的均值,无任何降维。}
\end{figure*}

\subsection{同质环境中角色的演化与涌现}
\begin{figure*}
  \includegraphics[width=\linewidth]{figures/evolution/evolution_10m_vs_11m.png}
  \caption{角色在$\mathtt{10m\_vs\_11m}$中的演化与涌现}\label{fig:role_evolution-homo}
\end{figure*}


\section{实验配置细节}\label{sec:exp-detail}

\section{本章小结}

