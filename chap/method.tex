% !TeX root = ../main.tex

\chapter{基于角色涌现的多智能体强化学习}
本章详细描述具体的基于角色的框架,以及考虑到“角色”应该具有的诸多性质,来设计损失函数,以引导“角色”涌现的完成。本章的结构为,首先介绍角色是什么,然后角色应该具有的四种性质,再然后引入让每个智能体的效用函数都基于自己的角色的方法,进一步再根据角色的性质设计用于优化的损失函数,最后再对总的损失函数进行总结,并给出本项目算法的框架图。同时,为了让本框架更加容易理解,本章在后面还附加上另一种从深度学习角度的原理阐述。

\section{角色的定义}
为了不引起混淆,简要描述一下本文使用的角色的定义。

本文的角色是对某一行为模式的抽象。它应该具有如下的表现特征:

i) \textbf{具有某一角色的智能体常常专注于某一个或某一些子任务;}

ii) \textbf{具有相似角色的智能体会有相似的行为;}

拿蚂蚁来举例,一个蚂蚁群通常分为如下的角色:工蚁、兵蚁、蚁后等,具有工蚁角色的蚂蚁通常的行为是采集食物、建造巢穴、哺育幼蚁等,如果两只蚂蚁具有相似的角色(比如都是工蚁),那么它们的行为就会相似,具体来说,在蚂蚁这个案例里,这两只蚂蚁的每日的轨迹可能是重叠的。但是反过来,不同的角色,比如工蚁和兵蚁,它们的行为轨迹就会大不相同。

最后,角色本身是复杂的概念,很难做一个完整的定义,本文只关注那些比较明显的特征。


\section{角色具有的性质}
本小节介绍本项目认为的角色应该具有的性质。

从社会学的角度来说,角色应该会拥有很复杂、广泛的性质,比如某个角色可能会在与其他角色的交互中不断演化、越来越复杂。但是完全表达这些复杂的性质是没有必要的,本项目的目的是为了利用角色这一概念中对多智能体强化学习有启发的部分,因此,考虑到有效性和可行性,角色应该要具有如下的性质:

i) \textbf{可识别性(Identifiable)}: 一个智能体的角色能够通过它的行为模式识别出来。这一性质意味着,一个角色应该是与行为模式挂钩的,这一点正符合对角色的定义,即“角色是对某一行为模式的抽象”。

ii) \textbf{专业化(Specialized)}: 具有相似角色的智能体应该专门处理相似的子任务。这一性质同时也意味着处理相似的子任务的智能体应该也是相似的角色。

iii) \textbf{动态(Dynamic)}: 一个智能体的角色应该能动态地变化,并且这种变化是为了适应环境。正如上文所说,角色会随着和环境、其他角色的交互不断演化。

iv) \textbf{通用性(Versatile)}: 涌现出来的角色应该足够不同,这样才能处理那些需要不同角色的任务。比如蚁群的生存繁衍任务,就需要工蚁、兵蚁、蚁后等角色才能完成。


\section{基于角色的智能体}
在表达上一节的性质之前,需要先设计如何让一个智能体基于角色。

注意到,在现在的多智能体系统中,每个智能体$i$都有一个局部的效用函数(或者一个局部的策略函数),为了让智能体是基于自己的角色的,让该智能体的局部效用(策略)函数的参数$\theta_i$决定于它自己的角色$\rho_i$. 

同时,为了让学到的角色拥有上面表述的性质,将角色编码到随机嵌入空间(stochastic embedding space), 在这个基础上,让每个智能体$i$的角色$\rho_i$采样自一个多维高斯分布,即$\mathcal{N}(\bm{\mu}_{\rho_i}, \bm{\sigma}_{\rho_i})$.



\section{角色的性质及损失函数设计}
\subsection{角色的动态变化性质}

\subsection{角色的可识别性和通用性}

\subsection{角色的专业化性质}

\subsection{总损失函数}

\section{框架图}

\section{其他阐释方式}
参数共享的解释

\section{本章小结}

