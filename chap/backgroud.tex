% !TeX root = ../main.tex

\chapter{基础概念与基本算法}
\section{单智能体强化学习简介}
本节会简单介绍一些必须的单智能体强化学习知识,然后下一节讲述本文用到的多智能体强化学习的建模。

\subsection{MDP}
强化学习的智能体需要在与环境的交互中做连续的决定。而这个环境常常表示成Markov决策过程(Markov Decision Process, MDP). MDP是定义为一个元组$(S,A,P,R,\gamma)$, 其中$S$和$A$分别表示状态和动作空间,$P:S\times A\to \Delta (S)$表示从状态$s\in S$采取某个动作$a\in A$后转移到$s'\in S$的概率,$R: S\times A\times S\to \mathbb{R}$是激励函数,表示智能体在这种转移的过程中获得的激励值,$\gamma$成为折扣值,表示智能体更加期待立即的激励而不是未来的激励。

在每个时刻$t$, 智能体面对的是系统的状态$s_t$, 然后做出一步动作$a_t$, 这会让系统转移到下一个环境$s_{t+1}\sim P(\cdot|s_t,a_t)$. 除此之外,智能体还会收到一个及时的激励$R(s_t,a_t,s_{t+1})$. 解决这个MDP的目标就是找到一个策略$\pi:S\to \Delta(A)$, 即一个从状态空间$S$到动作空间$A$的映射,也就是动作$a_t\sim \pi(\cdot|s_t)$, 最终的折扣后的累计激励为
\begin{equation}
    \mathbb{E}\left[\sum_{t\ge 0}\gamma^t R(s_t,a_t,s_{t+1})\bigg|a_t\sim\pi(\cdot|s_t),s_0    \right]
\end{equation}
相应地,可以定义在策略$\pi$下的Q函数和值函数:
\begin{align}
Q_\pi(s,a) &= \mathbb{E}\left[\sum_{t\ge 0}\gamma^t R(s_t,a_t,s_{t+1})\bigg|a_t\sim\pi(\cdot|s_t),a_0=a,s_0 =s\right] \\
V_\pi(s) &= \mathbb{E}\left[\sum_{t\ge 0}\gamma^t R(s_t,a_t,s_{t+1})\bigg|a_t\sim\pi(\cdot|s_t),s_0 =s\right]
\end{align}

\subsection{基于值函数的方法}
基于值函数的方法需要找到一个对Q函数的好的估计,比如最优Q函数$Q_{\pi^*}$, 然后每一步的动作就可以选取让这个Q值最大的$a$. 一个非常注明的基于值函数的算法是Q-Learning, 在这个算法里,只能以会维护一个Q值的估计$\hat{Q}(s,a)$. 当从状态$s$由动作$a$转移到下一状态$s'$时,智能体会收到一个激励$r$, 然后按照如下方式更新Q函数:
\begin{equation}
    \hat{Q}(s,a)\leftarrow (1-\alpha)\hat{Q}(s,a) + \alpha \left[ r + \gamma \max_{a'}\hat{Q}(s',a') \right]
\end{equation}
这里的$\alpha>0$是学习速率。

\subsection{基于策略函数的方法}
另一类方法是直接在策略函数空间搜索,通常这个策略函数用神经网络来估计,也就是$\pi_\theta(\cdot|s)\approx \pi(\cdot|s)$. 所以,最直接的方法就是好沿着增大累加激励的方向更新策略函数,也就是策略梯度方法:
\begin{equation}
    \nabla J(\theta) = \mathbb{E}_{a\sim \pi_{theta}(\cdot|s),s\sim\eta_{\pi_\theta}(\cdot)}\left[Q_{\pi_\theta}(s,a)\nabla \log \pi_\theta (a|s) \right]
\end{equation}
其中,$J(\theta)$是期望返回值,$Q_{\pi_\theta}$是在策略$\pi_\theta$下的Q函数,$\eta_{\pi_\theta}$是状态出现的频率。

\section{多智能体强化学习简介}
和单智能体强化学习类似,MARL也是考虑解决一个连续的决定问题,但是有多个智能体参与进来。具体来说,整个环境系统的演化和智能体收到的激励都会受到所有智能体的动作构成的联合动作的影响。

本项目考虑一个,完全协作的多智能体任务,这可以被建模成Dec-POMDP, 也就是~\cite{oliehoek2016concise}$G=\langle I, S, A, P, R, \Omega, O, n, \gamma\rangle$, 其中$A$是有限的动作空间, $I\equiv\{1,2,...,n\}$是有限的智能体集合, $\gamma\in[0, 1)$是discount factor,$s\in S$是环境真正的状态。这里考虑一个部分可观测的环境,智能体$i$只能观测到$o_i\in \Omega$, 这个观测来自观测函数$O(s, i)$. 每个智能体有个历史轨迹$\tau_i\in \Tau\equiv(\Omega\times A)^*$. 每个时间步,每个智能体$i$会选择一个动作$a_i\in A$, 和其他所有智能体的动作联合形成一个联合动作$\va$ $\in A^n$,然后根据转移函数$P(s'|s, \va)$会得到下一个状态$s'$,以及一个共享的激励$r=R(s,\va)$.联合策略函数$\bm{\pi}$能推导出一个联合的动作-值函数$Q_{tot}^{\bm{\pi}}(s,\va)=\mathbb{E}_{s_{0:\infty},\va_{0:\infty}}[\sum_{t=0}^\infty \gamma^{t}r_t|s_0=s,\va_0=\va,\bm{\pi}]$. 

为了更高效地学习智能体的策略,集中训练,分离执行(Centralized Training with Decentralized Execution (CTDE)~\cite{foerster2016learning, foerster2018counterfactual}被广泛应用。一种很有前景的利用CTDE的方式是值函数分解~\cite{sunehag2018value, rashid2018qmix, son2019qtran}, 简要来说就是每个智能体学习一个分离的局部效用函数,然后用一个混合网络来结合这些局部的值,输出全局的动作-值信息。本项目也是利用CTDE框架,不同的是,局部效用函数不是完全共享的,也不是完全各自独立的,而是在相似角色的智能体之间共享。

\section{星际争霸II实验环境简介}
星际争霸II单元微操作任务~\cite{samvelyan2019starcraft}被认为是最具挑战性的协同多智能体基准测试环境,因为它它有很高的控制复杂度和环境随机性。

许多算法\cite{foerster2017stabilising, foerster2018counterfactual, rashid2018qmix, mahajan2019maven}都是使用该环境训练测试的,本框架也是如此。

其框架\footnote{\url{https://github.com/oxwhirl/pymarl}}可以开源获取。

% TODO: 更多的关于sc2的东西,附图。thesis里有一点。

\section{相关工作与对比}
角色的涌现已经在很多文献中有记录了,比如蜜蜂~\cite{jeanson2005emergence}, 蚂蚁~\cite{gordon1996organization}, 人类~\cite{butler2012condensed}等等。在这些系统中,角色和分工紧密相连,并且对提升劳动效率很有作用。同时,很多的多智能体系统受到了这些自然界的系统的启发。它们会分解任务,让有相同角色的智能体专注于某个子任务,以此来降低设计的复杂度~\cite{wooldridge2000gaia, omicini2000soda, padgham2002prometheus, pavon2003agent, cossentino2005passi, zhu2008role, spanoudakis2010using, deloach2010mase, bonjean2014adelfe}. 这些方法都被设计用于有清晰结构的任务,比如软件工程~\cite{bresciani2004tropos}. 因此,它们会预定义角色以及相关的子任务~\cite{ Lhaksmana2018role}。但是本框架是隐式地把角色概念引入多智能体的连续决定,并且环境是动态和未知的。

深度多智能体强化学习在近几年有很大的进步。COMA~\citep{foerster2018counterfactual}, MADDPG~\citep{lowe2017multi}, PR2~\citep{wen2019probabilistic}, 和MAAC~\cite{iqbal2019actor}集中注意力在多智能体的策略梯度方法。另一条主线是关注基于值函数的多智能体强化学习, VDN~\citep{sunehag2018value}, QMIX~\citep{rashid2018qmix}, 和 QTRAN~\citep{son2019qtran}依次扩大了混合网络能表达的函数的范围。另一方面,涌现也是一个在深度多智能体强化学习逐渐引起重视的话题,相关的工作有交流的涌现~\cite{foerster2016learning, lazaridou2017multi, das2017learning, mordatch2018emergence}, 公平性的涌现~\cite{jiang2019learning}, 工具使用的涌现~\cite{baker2020emergent}, 这些涌也提供了一个用深度学习理解自然或者人工多智能体系统的新角度、

为了学到多样化以及可识别的角色,本算法提出又换关于个人角色和轨迹的互信息,最近的一个研究多智能体探索的算法MAVEN~\cite{mahajan2019maven}也用了类似的目标函数。但是不同的是,MAVEN的目标是集体的探索,这是在高层次思想的不同,同样也造成了很多技术细节的不同。除此之外,在实验章节,也比较了本算法与MAVEN的性能不同。

\section{本章小结}
本章先简单介绍了单智能体强化学习的背景,然后着重介绍本文所用的多智能体强化学习的建模,进一步介绍了本项目所采用的实验环境星际争霸II,最后给出了与本文相关的工作,以及本项目与之的不同。
